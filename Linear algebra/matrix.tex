\documentclass[12pt,a4paper]{article}
\usepackage{amsmath,amsthm,amssymb} 
\usepackage{mathtext} 
\usepackage[T1,T2A]{fontenc} 
\usepackage[utf8]{inputenc} 
\usepackage[english,russian]{babel}
\setlength{\parskip}{6pt}
\setlength{\parindent}{0ex}

\begin{document}

\section{Матрицы}
\subsection{Определитель}
Определитель 2-го порядка:
$$\begin{vmatrix}
1& 2 \\
3& 4
\end{vmatrix}
= 1 * 4 - 2 * 3 = -2$$
Определитель 3-го порядка:
$$\begin{pmatrix}
1& 2& 3 \\
4& 5& 6 \\
7& 8& 9
\end{pmatrix}
= 1*5*9 + 4*8*3 + 2*6*7 - 3*5*7 - 8*6*1 - 4*2*9 = 0$$
$$\begin{pmatrix}
2& 5& 7 \\
6& 3& 4 \\
5& -2& -3
\end{pmatrix}
= 2*3*(-3) + 6*(-2)*7 + 5*4*5 - 7*3*5 - 4*(-2)*2 - 6*5*(-3) = -1$$
Если определитель матрицы = 0, то обратной матрицы не существует.
\subsection{Транспонированная матрица}
$$A=\begin{pmatrix}
1& 2& 3 \\
4& 5& 6 \\
7& 8& 9
\end{pmatrix}$$
$$A^{T}=\begin{pmatrix}
1& 4& 7 \\
2& 5& 8 \\
3& 6& 9
\end{pmatrix}$$
\subsection{Миноры и алгебраическое дополнение}
Минор ($M_{строка, столбец}$):
$$A=\begin{pmatrix}
1& 2& 3 \\
4& 5& 6 \\
7& 8& 9
\end{pmatrix}$$
$$M_{11}=\begin{vmatrix}
5& 6 \\
8& 9
\end{vmatrix}$$
Алгебраическое дополнение ($A_{строка, столбец}$):
$$A_{11}=(-1)^{1+1}*M_{11}=1*(-3)=-3$$
\subsection{Обратная матрица}
1. Найти определитель матрицы $|A|$. Если определитель = 0, то обратной матрицы не существует.

2. Найти все алгебраические дополнения $A_{i,j}$. 

3. Составить вспомогательную матрицу $C$, состоящую из алгебраических дополнений. 

4. Найти матрицу $\tilde{A}$ путём траспонирования матрицы $C$. 

5. Найти обратную матрицу $$A^{-1}=\frac{1}{|A|}*\tilde{A}$$

Пример:
$$A=\begin{pmatrix}
1& 2& 3 \\
4& 5& 6 \\
7& 8& 0
\end{pmatrix}$$
1. $|A|=96+84-105-48=27$ \\
2. $A_{1,1}=(-1)^{1+1}*\begin{vmatrix}
5& 6 \\
8& 0
\end{vmatrix}=-48$\\
$A_{1,2}=(-1)^{1+2}*\begin{vmatrix}
4& 6 \\
7& 0
\end{vmatrix}=42$ и т.д. 

3. $C=\begin{pmatrix}
-48& 42& -3 \\
24& -21& 6 \\
-3& 6& -3
\end{pmatrix}$ 

4. $\tilde{A}=C^{T}=\begin{pmatrix}
-48& 24& -3 \\
42& -21& 6 \\
-3& 6& -3
\end{pmatrix}$ 

5. $A^{-1}=\frac{1}{|A|}*\tilde{A}=\frac{1}{27}*\begin{pmatrix}
-48& 24& -3 \\
42& -21& 6 \\
-3& 6& -3
\end{pmatrix}=\begin{pmatrix}
-\frac{16}{9}& \frac{8}{9}& -\frac{1}{9} \\
\frac{14}{9}& -\frac{7}{9}& \frac{2}{9} \\
-\frac{1}{9}& \frac{2}{9}& -\frac{1}{9}
\end{pmatrix}$
\section{Системы линейных уравнений}
\subsection{Метод Крамера}
\begin{equation*}
 \begin{cases}
   a_{1}x_{1}+b_{1}x{2}+c_{1}x_{3}=s_{1}
   \\
   a_{2}x_{1}+b_{2}x{2}+c_{2}x_{3}=s_{2}
   \\
   a_{3}x_{1}+b_{3}x{2}+c_{3}x_{3}=s_{3}
 \end{cases}
\end{equation*}

1. Найти определитель $D = \begin{vmatrix}
a_{1}& b_{1}& c_{1} \\
a_{2}& b_{2}& c_{2} \\
a_{3}& b_{3}& c_{3}
\end{vmatrix}$ \\
Если определитель = 0, то то система имеет бесконечно много решений или несовместна (не имеет решений). В этом случае правило Крамера не поможет, нужно использовать метод Гаусса. 

2. Если определитель $\neq$ 0, то система имеет единственное решение, и для нахождения корней мы должны вычислить еще три определителя:
$$D_{1} =
\begin{vmatrix}
s_{1}& b_{1}& c_{1} \\
s_{2}& b_{2}& c_{2} \\
s_{3}& b_{3}& c_{3}
\end{vmatrix} 
D_{2} =
\begin{vmatrix}
a_{1}& s_{1}& c_{1} \\
a_{2}& s_{2}& c_{2} \\
a_{3}& s_{3}& c_{3}
\end{vmatrix}
D_{3} =
\begin{vmatrix}
a_{1}& b_{1}& s_{1} \\
a_{2}& b_{2}& s_{2} \\
a_{3}& b_{3}& s_{3}
\end{vmatrix}$$
3. Корни уравнения находим по формулам:
$$x_{1} = \frac{D_{1}}{D}, x_{2} = \frac{D_{2}}{D}, x_{3} = \frac{D_{3}}{D}$$
Пример: \begin{equation*}
 \begin{cases}
   3x_{1}-2x_{2}+4x_{3}=21
   \\
   3x_{1}+4x_{2}-2x_{3}=9
   \\
   2x_{1}-x_{2}-x_{3}=10
 \end{cases}
\end{equation*}

1. $D = \begin{vmatrix}
3& -2& 4 \\
3& 4& -2 \\
2& -1& -1
\end{vmatrix} = -60$, значит, система имеет единственное решение. 

2. $D_{1} = \begin{vmatrix}
21& -2& 4 \\
9& 4& -2 \\
10& -1& -1
\end{vmatrix} = -300$ 
$x_{1} = \frac{D_{1}}{D} = \frac{-300}{-60} = 5$ 

$D_{2} = \begin{vmatrix}
3& 21& 4 \\
3& 9& -2 \\
2& 10& -1
\end{vmatrix} = 60$ 
$x_{2} = \frac{D_{2}}{D} = \frac{60}{-60} = -1$ 

$D_{3} = \begin{vmatrix}
3& -2& 21 \\
3& 4& 9 \\
2& -1& 10
\end{vmatrix} = -60$ 
$x_{3} = \frac{D_{3}}{D} = \frac{-60}{-60} = 1$ 

Ответ: $x_{1} = 5, x_{2} = -1, x_{3} = 1$.
\end{document}
