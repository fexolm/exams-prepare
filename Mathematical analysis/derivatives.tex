\documentclass[12pt,a4paper]{article}
\usepackage[T1,T2A]{fontenc}
\usepackage[utf8]{inputenc}
\usepackage[english,russian]{babel}
\parindent=1cm

\title{\LaTeX}  
\date{}  
\author{Арслан Валеев}

\begin{document}  

\section{Производные}
\subsection{Определение производной функции}

Производная функции характеризует скорость изменения функции в данной точке. Определяется как предел отношения приращения функции к приращению её аргумента при стремлении приращения аргумента к нулю, если такой предел существует.
\\\\
Функцию, имеющую конечную производную (в некоторой точке), называют дифференцируемой (в данной точке). Процесс вычисления производной называется дифференцированием.
\\\\
Функция, имеющая производную в точке, непрерывна в ней. Обратное не всегда верно.
Для производной используются обозначения:

$$f'(x)=y'(x)=\frac{df}{dx}=\frac{dy}{dx}$$

\subsection{Определение производной функции через предел}

$$f'(x_0)
= \lim_{x \to x_0}\frac{f(x)-f(x_0)}{x-x_0}
= \lim_{\Delta x \to 0}\frac{f(x_0+ \Delta x)-f(x_0)}{\Delta x}
= \lim_{\Delta x \to 0}\frac{\Delta f(x)}{\Delta x}
$$

\subsection{Дифференцируемость}

Производная $f'(x_0)$, будучи пределом, может не существовать или существовать и быть конечной или бесконечной. 
\\
Функция $f$ является дифференцируемой в точке $x_0$ тогда и только тогда, когда её производная в этой точке существует и конечна:

$$f \in D(x_0) \Leftrightarrow \exists f'(x_0) \in (-\infty ; \infty).$$

Для дифференцируемой в $x_0$ функции $f$ в окрестности $U(x_0)$ справедливо представление

$$f(x)=f(x_0)+f'(x_0)(x-x_0)+o(x-x_0), x \rightarrow x_0$$

\subsection{Примеры}

Потом

\subsection{Производные степенных функций}

$$(c)'=0$$
где c - константа
$$(x^a)'=ax^{a-1}$$
$$(a^x)'=a^x\ln a$$
$$(e^x)^{(n)}=e^x$$
$$(\log_{a} x)'=\frac{1}{x\ln a}$$

\subsection{Производные тригонометрических функций}

$$(\sin x)'=\cos x$$
$$(\cos x)'=-\sin x$$
$$(\tg x)'=\frac{1}{\cos^2 x}$$
$$(\ctg x)'=-\frac{1}{\sin^2 x}$$

\subsection{Производные обратных тригонометрических функций}

$$(\arcsin x)'=\frac{1}{\sqrt{1-x^2}}$$
$$(\arccos x)'=-\frac{1}{\sqrt{1-x^2}}$$
$$(\arctg x)'=\frac{1}{1+x^2}$$
$$(\arcctg x)'=-\frac{1}{1+x^2}$$

\subsection{Примеры для самостоятельного решения}

Потом

\end{document}