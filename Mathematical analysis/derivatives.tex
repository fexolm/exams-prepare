\documentclass[12pt,a4paper]{article}
\usepackage[T1,T2A]{fontenc}
\usepackage[utf8]{inputenc}
\usepackage[english,russian]{babel}
\parindent=1cm

\title{\LaTeX}  
\date{}  
\author{Арслан Валеев}

\begin{document}  

\section{Производные}
\subsection{Определение производной функции}

Производная функции характеризует скорость изменения функции в данной точке. Определяется как предел отношения приращения функции к приращению её аргумента при стремлении приращения аргумента к нулю, если такой предел существует.
\\\\
Функцию, имеющую конечную производную (в некоторой точке), называют дифференцируемой (в данной точке). Процесс вычисления производной называется дифференцированием.
\\\\
Функция, имеющая производную в точке, непрерывна в ней. Обратное не всегда верно.
Для производной используются обозначения:

$$f'(x)=y'(x)=\frac{df}{dx}=\frac{dy}{dx}$$

\subsection{Определение производной функции через предел}

$$f'(x_0)
= \lim_{x \to x_0}\frac{f(x)-f(x_0)}{x-x_0}
= \lim_{\Delta x \to 0}\frac{f(x_0+ \Delta x)-f(x_0)}{\Delta x}
= \lim_{\Delta x \to 0}\frac{\Delta f(x)}{\Delta x}
$$

\subsection{Примеры}

\subsubsection{Пример с корнем$\;f(x)=\sqrt{x}$}
$$f'(x)=\frac{1}{2\sqrt{x}}$$
\subsubsection{Производная суммы равна сумме производных}
$$f(x)=6+x+3x^2-\sin{x}-2\sqrt[3]{x}+\frac{1}{x^2}-11\ctg{x}$$
$$f'(x)=(6+x+3x^2-\sin{x}-2\sqrt[3]{x}+\frac{1}{x^2}-11\ctg{x})'$$
$$f'(x)= 6'+x'+(3x^2)'-(\sin{x})'-(2\sqrt[3]{x})'+(\frac{1}{x^2})'-(11\ctg{x})'$$
$$f'(x)= 0+1+6x-\cos{x}-\frac{2}{3}\sqrt[3]x^2-\frac{2}{x^3}+11\sin^2{x}$$
\subsubsection{Производная произведения функций$\;(uv)'=u'v+uv'$}
$$f(x)=x^3\arcsin{x}$$
$$f'(x)=(x^3\arcsin{x})'$$
$$f'(x)=(x^3)'\arcsin{x}+x^3(\arcsin{x})'$$
$$f'(x)=3x^2\arcsin{x}+x^3*\frac{1}{\sqrt{1-x^2}}$$
\subsubsection{Производная частного функций$\;(\frac{u}{v})'=\frac{u'v-uv'}{v^2}$}
$$f(x)=\frac{2(3x-4)}{x^2+1}$$
$$f'(x)=2(\frac{3x-4}{x^2+1})'$$
$$f'(x)=2(\frac{(3x-4)'(x^2+1) - (3x-4)(x^2+1)'}{(x^2+1)^2})$$
$$f'(x)=2(\frac{3*(x^2+1)-(3x-4)*2x}{(x^2+1)^2})$$
$$f'(x)=\frac{2(-3x^2+8x+3)}{(x^2+1)^2}$$
\subsubsection{Производная сложной функции$\;(u(v))'=u'(v)*v'$\\Правило дифференцирования сложной функции применяется в последнюю очередь!}
$$f(x)=\sin{(3x-5)}$$
$$f'(x)=(\sin{(3x-5)})'$$
$$f'(x)=\cos{(3x-5)}*(3-5)$$
$$f'(x)=-2\cos{(3x-5)}$$

\subsection{Производные степенных функций}

$$(c)'=0$$
где c - константа
$$(x^a)'=ax^{a-1}$$
$$(a^x)'=a^x\ln a$$
$$(e^x)^{(n)}=e^x$$
$$(\log_{a} x)'=\frac{1}{x\ln a}$$

\subsection{Производные тригонометрических функций}

$$(\sin x)'=\cos x$$
$$(\cos x)'=-\sin x$$
$$(\tg x)'=\frac{1}{\cos^2 x}$$
$$(\ctg x)'=-\frac{1}{\sin^2 x}$$

\subsection{Производные обратных тригонометрических функций}

$$(\arcsin x)'=\frac{1}{\sqrt{1-x^2}}$$
$$(\arccos x)'=-\frac{1}{\sqrt{1-x^2}}$$
$$(\arctg x)'=\frac{1}{1+x^2}$$
$$(\arcctg x)'=-\frac{1}{1+x^2}$$

\end{document}