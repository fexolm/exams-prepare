\documentclass[12pt,a4paper]{article}
\usepackage[T1,T2A]{fontenc}
\usepackage[utf8]{inputenc}
\usepackage[english,russian]{babel}
\usepackage{amsmath}
\usepackage{graphicx}
\graphicspath{{pictures/}}
\DeclareGraphicsExtensions{.pdf,.png,.jpg}

\title{\LaTeX}  
\date{}  
\author{Арслан Валеев}
\setlength{\parskip}{6pt}
\setlength{\parindent}{0ex}

\begin{document}

\section{Ряды}
\subsection{Необходимый признак сходимости}
\[ \lim_{x \to \infty}a_n=0 \]
Пример.
\[ 
\lim_{x \to \infty}a_n
=\lim_{x \to \infty}(2n-1)
=\infty \neq 0
\]
Однако, если выполняется необходимое условие, ряд может как сходиться, так и расходиться.
\subsection{Гармонический ряд}
Несмотря на то, что его предел = 0, он расходится.
\[ \sum_{n=1}^{\infty} \frac{1}{n} \]
Общий вид гармонического ряда.
\[ \sum_{n=1}^{\infty} \frac{1}{n^a} \]
Расходится, если $ a<=1 $.
Сходится, если $ a>1 $.
\subsection{Достаточные признаки сходимости}
В разделе будут рассматриваться только ПОЛОЖИТЕЛЬНЫЕ числовые ряды.
\subsubsection{Признак сравнения}
Если сходится $ \sum_{n=1}^{\infty} b_n $, и, начиная с какого-то $ n $ выполняется $ a_n <= b_n $, то $ \sum_{n=1}^{\infty} a_n $ тоже сходится.
\subsubsection{Предельный признак сравнения}
Применяется, когда: \\
1) В знаменателе находится многочлен.\\
2) Многочлены находятся и в числителе и в знаменателе.\\
3) Один или оба многочлена могут быть под корнем.\\
4) Многочленов и корней, разумеется, может быть и больше.\\

Ряды $ \sum_{n=1}^{\infty} a_n $ и $ \sum_{n=1}^{\infty} b_n $ сходятся или расходятся одновременно, если
\[ \lim_{n \to \infty} \frac{a_n}{b_n} \neq 0 \]
\subsubsection{Признак Даламбера}
Применяется, когда: \\
1) В общий член ряда входит число в степени. Неважно, где оно располагается, в числителе или в знаменателе.\\
2) В общий член ряда входит факториал. Факториал может располагаться вверху или внизу дроби.\\
3) Если в общем члене ряда есть «цепочка множителей». \\
p.s. Вместе со степенями или факториалами в часто встречаются многочлены, но это ничего не меняет, надо юзать Даламбера.

Есть ряд $ \sum_{n=1}^{\infty} a_n $. Если существует
\[ \lim_{n \to \infty} \frac{a_{n+1}}{a_n}=D \]
а) При $ D<1 $ ряд сходится.\\
б) При $ D>1 $ ряд расходится.\\
в) При $ D=1 $ признак на дает ответа.\\
p.s. Чаще всего единица получается в том случае, когда признак Даламбера пытаются применить там, где нужно использовать предельный признак сравнения.
\subsubsection{Радикальный признак сходимости Коши}
Радикальный признак Коши обычно использует когда общий член ряда ПОЛНОСТЬЮ находится в степени, зависящей от $ n $.

Рассматриваем $ \sum_{n=1}^{\infty} a_n $. Если существует предел:
\[ \lim_{n \to \infty} \sqrt[n]{a_n}=D \]
а) При $ D<1 $ ряд сходится.\\
б) При $ D>1 $ ряд расходится.\\
в) При $ D=1 $ признак на дает ответа.\\
\subsubsection{Интегральный признак сходимости Коши}
Основной предпосылкой использования интегрального признака Коши является то, что в общем члене ряда содержатся множители, похожие на некоторую функцию и её производную.

Рассматриваем $ \sum_{n=1}^{\infty} a_n $. Если существует несобственный интеграл:
\[ \int_{1}^{\infty}a_xdx \]
то ряд сходится или расходится вместе с этим интегралом.
\subsection{Примеры}
todo
\end{document}
