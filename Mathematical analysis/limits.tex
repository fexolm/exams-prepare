\documentclass[12pt,a4paper]{article}
\usepackage[T1,T2A]{fontenc}
\usepackage[utf8]{inputenc}
\usepackage[english,russian]{babel}

\title{\LaTeX}  
\date{}  
\author{Артём Алексеев}
\setlength{\parskip}{6pt}
\setlength{\parindent}{0ex}

\begin{document}  

\section{Пределы}
\subsection{Простые пределы}

Сначала пытаемся подставить значение в формулу
$$\lim_{x \to 1}\frac{2x^2-3x-5}{x+1} = \frac{2*1^2-3*1-5}{1+1} = \frac{-6}{2} = -3$$

С нулем или $\infty$ поступаем так же
$$\lim_{x \to \infty}1+x = 1 + \infty = \infty$$

\subsection{Неопределенность вида $\frac{\infty}{\infty}$}

Такая неопределенность возникает, когда х встречается и в числителе и в знаменателе, при этом $x \to \infty$.
Решаются такие примеры путем деления чистителя и знаменателя на х в старшей степени.

$$\lim_{x \to \infty}\frac{2x^2-3x-5}{1+x+3x^2} = \frac{\infty}{\infty}$$

Разделим числитель и знаменатель на $x^2$.

$$(*) 
= \lim_{x \to \infty}\frac{\frac{2x^2-3x-5}{x^2}}{\frac{1+x+3x^2}{x^2}}
= \lim_{x \to \infty}\frac{\frac{2x^2}{x^2}-\frac{3x}{x^2}-\frac{5}{x^2}}{\frac{1}{x^2} + \frac{x}{x^2} + \frac{3x^2}{x^2}}
= \lim_{x \to \infty}\frac{2-\frac{3}{x}^{\to 0}-\frac{5}{x^2}^{\to 0}}{\frac{1}{x^2}^{\to 0} + \frac{1}{x}^{\to 0} + 3}
= \frac{2}{3}
$$

\subsection{Неопределенность вида $\frac{0}{0}$}

$$\lim_{x \to -1}{2x^2-3x-5}{x+1}$$

Подставим -1 в дробь.

$$\frac{2(-1)^2-3(-1)-5}{-1+1}=\frac{0}{0}$$

Получаем неопределенность вида $\frac{0}{0}$. 

Для того чтобы решить такую неопределенность нужно разложить числитель и знаменатель на множители.

$$ D = (-3)^2-4*2*(-5) = 9+40 = 49 $$
$$ \sqrt[2]{D} = \sqrt[2]{49} = 7 $$
$$ 2x^2-3x-5=2(x-(-1))*(x-\frac{5}{2})=2(x+1)*(x-\frac{5}{2})=(x+1)*(2x-5) $$

Вернемся к нашему уравнению

$$ (*) = \lim_{x \to -1}\frac{(x+1)*(2x-5)}{x+1} = \lim_{x \to -1}(2x-5) = 2 * (-1) - 5 = -7 $$

\subsection{Неопределенность вида $\frac{0}{0}$ (Метод умножения на сопряженное)}
Данный метод применяется, когда неопределенность вида $\frac{0}{0}$ появляется в выражениях содержащих корни.

Рассмотрим пример
$$ \lim_{x \to 3}\frac{\sqrt[2]{x+6}-\sqrt[2]{10x-21}}{5x-15} $$

Подставим 3 в уравнение, получим
$$ \frac{\sqrt[2]{3+6}-\sqrt[2]{10*3-21}}{5*3-15} = \frac{\sqrt[2]{9}-\sqrt[2]{9}}{15-15} = \frac{0}{0} $$

Получена неопределенность $\frac{0}{0}$.
Умножим числитель и знаменатель на сопряженное числителю, получим
$$ (*) = \lim_{x \to 3}\frac{(\sqrt[2]{x+6}-\sqrt[2]{10x-21}) * (\sqrt[2]{x+6}+\sqrt[2]{10x-21})}{(5x-15) * (\sqrt[2]{x+6}+\sqrt[2]{10x-21})} $$
$$ = \lim_{x \to 3}\frac{(\sqrt[2]{x+6})^2 - (\sqrt[2]{10x-21})^2}{(5x-15) * (\sqrt[2]{x+6}+\sqrt[2]{10x-21})} $$
$$ = \lim_{x \to 3}\frac{x+6-10x+21}{(5x-15) * (\sqrt[2]{x+6}+\sqrt[2]{10x-21})} $$
$$ = \lim_{x \to 3}\frac{-9x+27}{(5x-15) * (\sqrt[2]{x+6}+\sqrt[2]{10x-21})} $$

Из разности корней получили сумму, поэтому можем подставить в корни 3
$$ (*) = \lim_{x \to 3}\frac{-9x+27}{(5x-15) * (\sqrt[2]{3+6}+\sqrt[2]{10 * 3 -21})} $$
$$ = \lim_{x \to 3}\frac{-9x+27}{(5x-15) * (3+3} = \lim_{x \to 3}\frac{-9x+27}{(5x-15) * 6}
= \frac{1}{6} \lim_{x \to 3}\frac{-9x+27}{5x-15} $$

Теперь как было описано ранее раскладываем числитель и знаменатель на множители
$$ (*) = \frac{1}{6} * \frac{-9 * (x-3)}{5*(x-3)} = \frac{1}{6} * (-\frac{9}{5}) = \frac{-3}{10}$$
\subsection{Правило Лопиталя}
Правило лопиталя позволяет избавиться как от неопределенносей вида $\frac{0}{0}$ так и от неопределенностей вида $\frac{\infty}{\infty}$.

Пусть функции f(x) и g(x) бесконечно малы (или бесконечно велики) в некоторой точке k, если существует предел из отношений $\lim_{x \to k}\frac{f(x)}{g(x)}$, то $\lim_{x \to k}\frac{f(x)}{g(x)} = \lim_{x \to k}\frac{f'(x)}{g'(x)}$

Пример:
$$ \lim_{x \to 2}\frac{8-2x^2}{x^2+4x-12}=\frac{0}{0}=\lim_{x \to 2}\frac{(8-2x^2)'}{(x^2+4x-12)'}$$ 
$$=\lim_{x \to 2}\frac{0-4x}{2x+4-0} = \frac{-4 * 2}{2 * 2 + 4} = \frac{-8}{8} = -1$$
\subsection{Замечательные пределы}
\subsubsection{Первый замечательный предел}
$$ \lim_{\alpha \to 0} \frac{\sin \alpha}{\alpha} = 1 $$
\subsubsection{Второй замечательный предел} 
$$ \lim_{\alpha \to \infty}(1 + \frac{1}{\alpha})^\alpha = e $$
\subsection{Примеры для самостоятельного решения}
TODO: написать примеры
\end{document}
